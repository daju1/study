\documentclass[11pt]{article}

    \usepackage[T2A]{fontenc}
\usepackage[utf8]{inputenc}
\usepackage[english, russian]{babel}

\usepackage[breakable]{tcolorbox}
    \usepackage{parskip} % Stop auto-indenting (to mimic markdown behaviour)
    
    \usepackage{iftex}
    \ifPDFTeX
    	\usepackage[T1]{fontenc}
    	\usepackage{mathpazo}
    \else
    	\usepackage{fontspec}
    \fi

    % Basic figure setup, for now with no caption control since it's done
    % automatically by Pandoc (which extracts ![](path) syntax from Markdown).
    \usepackage{graphicx}
    % Maintain compatibility with old templates. Remove in nbconvert 6.0
    \let\Oldincludegraphics\includegraphics
    % Ensure that by default, figures have no caption (until we provide a
    % proper Figure object with a Caption API and a way to capture that
    % in the conversion process - todo).
    \usepackage{caption}
    \DeclareCaptionFormat{nocaption}{}
    \captionsetup{format=nocaption,aboveskip=0pt,belowskip=0pt}

    \usepackage{float}
    \floatplacement{figure}{H} % forces figures to be placed at the correct location
    \usepackage{xcolor} % Allow colors to be defined
    \usepackage{enumerate} % Needed for markdown enumerations to work
    \usepackage{geometry} % Used to adjust the document margins
    \usepackage{amsmath} % Equations
    \usepackage{amssymb} % Equations
    \usepackage{textcomp} % defines textquotesingle
    % Hack from http://tex.stackexchange.com/a/47451/13684:
    \AtBeginDocument{%
        \def\PYZsq{\textquotesingle}% Upright quotes in Pygmentized code
    }
    \usepackage{upquote} % Upright quotes for verbatim code
    \usepackage{eurosym} % defines \euro
    \usepackage[mathletters]{ucs} % Extended unicode (utf-8) support
    \usepackage{fancyvrb} % verbatim replacement that allows latex
    \usepackage{grffile} % extends the file name processing of package graphics 
                         % to support a larger range
    \makeatletter % fix for old versions of grffile with XeLaTeX
    \@ifpackagelater{grffile}{2019/11/01}
    {
      % Do nothing on new versions
    }
    {
      \def\Gread@@xetex#1{%
        \IfFileExists{"\Gin@base".bb}%
        {\Gread@eps{\Gin@base.bb}}%
        {\Gread@@xetex@aux#1}%
      }
    }
    \makeatother
    \usepackage[Export]{adjustbox} % Used to constrain images to a maximum size
    \adjustboxset{max size={0.9\linewidth}{0.9\paperheight}}

    % The hyperref package gives us a pdf with properly built
    % internal navigation ('pdf bookmarks' for the table of contents,
    % internal cross-reference links, web links for URLs, etc.)
    \usepackage{hyperref}
    % The default LaTeX title has an obnoxious amount of whitespace. By default,
    % titling removes some of it. It also provides customization options.
    \usepackage{titling}
    \usepackage{longtable} % longtable support required by pandoc >1.10
    \usepackage{booktabs}  % table support for pandoc > 1.12.2
    \usepackage[inline]{enumitem} % IRkernel/repr support (it uses the enumerate* environment)
    \usepackage[normalem]{ulem} % ulem is needed to support strikethroughs (\sout)
                                % normalem makes italics be italics, not underlines
    \usepackage{mathrsfs}
    
\usepackage{wrapfig}
\usepackage[rightcaption]{sidecap}
\providecommand{\keywords}[1]{\textbf{\textit{Keywords:}} #1}

\author{A. Yu. Drozdov}

    
    % Colors for the hyperref package
    \definecolor{urlcolor}{rgb}{0,.145,.698}
    \definecolor{linkcolor}{rgb}{.71,0.21,0.01}
    \definecolor{citecolor}{rgb}{.12,.54,.11}

    % ANSI colors
    \definecolor{ansi-black}{HTML}{3E424D}
    \definecolor{ansi-black-intense}{HTML}{282C36}
    \definecolor{ansi-red}{HTML}{E75C58}
    \definecolor{ansi-red-intense}{HTML}{B22B31}
    \definecolor{ansi-green}{HTML}{00A250}
    \definecolor{ansi-green-intense}{HTML}{007427}
    \definecolor{ansi-yellow}{HTML}{DDB62B}
    \definecolor{ansi-yellow-intense}{HTML}{B27D12}
    \definecolor{ansi-blue}{HTML}{208FFB}
    \definecolor{ansi-blue-intense}{HTML}{0065CA}
    \definecolor{ansi-magenta}{HTML}{D160C4}
    \definecolor{ansi-magenta-intense}{HTML}{A03196}
    \definecolor{ansi-cyan}{HTML}{60C6C8}
    \definecolor{ansi-cyan-intense}{HTML}{258F8F}
    \definecolor{ansi-white}{HTML}{C5C1B4}
    \definecolor{ansi-white-intense}{HTML}{A1A6B2}
    \definecolor{ansi-default-inverse-fg}{HTML}{FFFFFF}
    \definecolor{ansi-default-inverse-bg}{HTML}{000000}

    % common color for the border for error outputs.
    \definecolor{outerrorbackground}{HTML}{FFDFDF}

    % commands and environments needed by pandoc snippets
    % extracted from the output of `pandoc -s`
    \providecommand{\tightlist}{%
      \setlength{\itemsep}{0pt}\setlength{\parskip}{0pt}}
    \DefineVerbatimEnvironment{Highlighting}{Verbatim}{commandchars=\\\{\}}
    % Add ',fontsize=\small' for more characters per line
    \newenvironment{Shaded}{}{}
    \newcommand{\KeywordTok}[1]{\textcolor[rgb]{0.00,0.44,0.13}{\textbf{{#1}}}}
    \newcommand{\DataTypeTok}[1]{\textcolor[rgb]{0.56,0.13,0.00}{{#1}}}
    \newcommand{\DecValTok}[1]{\textcolor[rgb]{0.25,0.63,0.44}{{#1}}}
    \newcommand{\BaseNTok}[1]{\textcolor[rgb]{0.25,0.63,0.44}{{#1}}}
    \newcommand{\FloatTok}[1]{\textcolor[rgb]{0.25,0.63,0.44}{{#1}}}
    \newcommand{\CharTok}[1]{\textcolor[rgb]{0.25,0.44,0.63}{{#1}}}
    \newcommand{\StringTok}[1]{\textcolor[rgb]{0.25,0.44,0.63}{{#1}}}
    \newcommand{\CommentTok}[1]{\textcolor[rgb]{0.38,0.63,0.69}{\textit{{#1}}}}
    \newcommand{\OtherTok}[1]{\textcolor[rgb]{0.00,0.44,0.13}{{#1}}}
    \newcommand{\AlertTok}[1]{\textcolor[rgb]{1.00,0.00,0.00}{\textbf{{#1}}}}
    \newcommand{\FunctionTok}[1]{\textcolor[rgb]{0.02,0.16,0.49}{{#1}}}
    \newcommand{\RegionMarkerTok}[1]{{#1}}
    \newcommand{\ErrorTok}[1]{\textcolor[rgb]{1.00,0.00,0.00}{\textbf{{#1}}}}
    \newcommand{\NormalTok}[1]{{#1}}
    
    % Additional commands for more recent versions of Pandoc
    \newcommand{\ConstantTok}[1]{\textcolor[rgb]{0.53,0.00,0.00}{{#1}}}
    \newcommand{\SpecialCharTok}[1]{\textcolor[rgb]{0.25,0.44,0.63}{{#1}}}
    \newcommand{\VerbatimStringTok}[1]{\textcolor[rgb]{0.25,0.44,0.63}{{#1}}}
    \newcommand{\SpecialStringTok}[1]{\textcolor[rgb]{0.73,0.40,0.53}{{#1}}}
    \newcommand{\ImportTok}[1]{{#1}}
    \newcommand{\DocumentationTok}[1]{\textcolor[rgb]{0.73,0.13,0.13}{\textit{{#1}}}}
    \newcommand{\AnnotationTok}[1]{\textcolor[rgb]{0.38,0.63,0.69}{\textbf{\textit{{#1}}}}}
    \newcommand{\CommentVarTok}[1]{\textcolor[rgb]{0.38,0.63,0.69}{\textbf{\textit{{#1}}}}}
    \newcommand{\VariableTok}[1]{\textcolor[rgb]{0.10,0.09,0.49}{{#1}}}
    \newcommand{\ControlFlowTok}[1]{\textcolor[rgb]{0.00,0.44,0.13}{\textbf{{#1}}}}
    \newcommand{\OperatorTok}[1]{\textcolor[rgb]{0.40,0.40,0.40}{{#1}}}
    \newcommand{\BuiltInTok}[1]{{#1}}
    \newcommand{\ExtensionTok}[1]{{#1}}
    \newcommand{\PreprocessorTok}[1]{\textcolor[rgb]{0.74,0.48,0.00}{{#1}}}
    \newcommand{\AttributeTok}[1]{\textcolor[rgb]{0.49,0.56,0.16}{{#1}}}
    \newcommand{\InformationTok}[1]{\textcolor[rgb]{0.38,0.63,0.69}{\textbf{\textit{{#1}}}}}
    \newcommand{\WarningTok}[1]{\textcolor[rgb]{0.38,0.63,0.69}{\textbf{\textit{{#1}}}}}
    
    
    % Define a nice break command that doesn't care if a line doesn't already
    % exist.
    \def\br{\hspace*{\fill} \\* }
    % Math Jax compatibility definitions
    \def\gt{>}
    \def\lt{<}
    \let\Oldtex\TeX
    \let\Oldlatex\LaTeX
    \renewcommand{\TeX}{\textrm{\Oldtex}}
    \renewcommand{\LaTeX}{\textrm{\Oldlatex}}
    % Document parameters
    % Document title
    \title{LW}
    
    
    
    
    
% Pygments definitions
\makeatletter
\def\PY@reset{\let\PY@it=\relax \let\PY@bf=\relax%
    \let\PY@ul=\relax \let\PY@tc=\relax%
    \let\PY@bc=\relax \let\PY@ff=\relax}
\def\PY@tok#1{\csname PY@tok@#1\endcsname}
\def\PY@toks#1+{\ifx\relax#1\empty\else%
    \PY@tok{#1}\expandafter\PY@toks\fi}
\def\PY@do#1{\PY@bc{\PY@tc{\PY@ul{%
    \PY@it{\PY@bf{\PY@ff{#1}}}}}}}
\def\PY#1#2{\PY@reset\PY@toks#1+\relax+\PY@do{#2}}

\@namedef{PY@tok@w}{\def\PY@tc##1{\textcolor[rgb]{0.73,0.73,0.73}{##1}}}
\@namedef{PY@tok@c}{\let\PY@it=\textit\def\PY@tc##1{\textcolor[rgb]{0.25,0.50,0.50}{##1}}}
\@namedef{PY@tok@cp}{\def\PY@tc##1{\textcolor[rgb]{0.74,0.48,0.00}{##1}}}
\@namedef{PY@tok@k}{\let\PY@bf=\textbf\def\PY@tc##1{\textcolor[rgb]{0.00,0.50,0.00}{##1}}}
\@namedef{PY@tok@kp}{\def\PY@tc##1{\textcolor[rgb]{0.00,0.50,0.00}{##1}}}
\@namedef{PY@tok@kt}{\def\PY@tc##1{\textcolor[rgb]{0.69,0.00,0.25}{##1}}}
\@namedef{PY@tok@o}{\def\PY@tc##1{\textcolor[rgb]{0.40,0.40,0.40}{##1}}}
\@namedef{PY@tok@ow}{\let\PY@bf=\textbf\def\PY@tc##1{\textcolor[rgb]{0.67,0.13,1.00}{##1}}}
\@namedef{PY@tok@nb}{\def\PY@tc##1{\textcolor[rgb]{0.00,0.50,0.00}{##1}}}
\@namedef{PY@tok@nf}{\def\PY@tc##1{\textcolor[rgb]{0.00,0.00,1.00}{##1}}}
\@namedef{PY@tok@nc}{\let\PY@bf=\textbf\def\PY@tc##1{\textcolor[rgb]{0.00,0.00,1.00}{##1}}}
\@namedef{PY@tok@nn}{\let\PY@bf=\textbf\def\PY@tc##1{\textcolor[rgb]{0.00,0.00,1.00}{##1}}}
\@namedef{PY@tok@ne}{\let\PY@bf=\textbf\def\PY@tc##1{\textcolor[rgb]{0.82,0.25,0.23}{##1}}}
\@namedef{PY@tok@nv}{\def\PY@tc##1{\textcolor[rgb]{0.10,0.09,0.49}{##1}}}
\@namedef{PY@tok@no}{\def\PY@tc##1{\textcolor[rgb]{0.53,0.00,0.00}{##1}}}
\@namedef{PY@tok@nl}{\def\PY@tc##1{\textcolor[rgb]{0.63,0.63,0.00}{##1}}}
\@namedef{PY@tok@ni}{\let\PY@bf=\textbf\def\PY@tc##1{\textcolor[rgb]{0.60,0.60,0.60}{##1}}}
\@namedef{PY@tok@na}{\def\PY@tc##1{\textcolor[rgb]{0.49,0.56,0.16}{##1}}}
\@namedef{PY@tok@nt}{\let\PY@bf=\textbf\def\PY@tc##1{\textcolor[rgb]{0.00,0.50,0.00}{##1}}}
\@namedef{PY@tok@nd}{\def\PY@tc##1{\textcolor[rgb]{0.67,0.13,1.00}{##1}}}
\@namedef{PY@tok@s}{\def\PY@tc##1{\textcolor[rgb]{0.73,0.13,0.13}{##1}}}
\@namedef{PY@tok@sd}{\let\PY@it=\textit\def\PY@tc##1{\textcolor[rgb]{0.73,0.13,0.13}{##1}}}
\@namedef{PY@tok@si}{\let\PY@bf=\textbf\def\PY@tc##1{\textcolor[rgb]{0.73,0.40,0.53}{##1}}}
\@namedef{PY@tok@se}{\let\PY@bf=\textbf\def\PY@tc##1{\textcolor[rgb]{0.73,0.40,0.13}{##1}}}
\@namedef{PY@tok@sr}{\def\PY@tc##1{\textcolor[rgb]{0.73,0.40,0.53}{##1}}}
\@namedef{PY@tok@ss}{\def\PY@tc##1{\textcolor[rgb]{0.10,0.09,0.49}{##1}}}
\@namedef{PY@tok@sx}{\def\PY@tc##1{\textcolor[rgb]{0.00,0.50,0.00}{##1}}}
\@namedef{PY@tok@m}{\def\PY@tc##1{\textcolor[rgb]{0.40,0.40,0.40}{##1}}}
\@namedef{PY@tok@gh}{\let\PY@bf=\textbf\def\PY@tc##1{\textcolor[rgb]{0.00,0.00,0.50}{##1}}}
\@namedef{PY@tok@gu}{\let\PY@bf=\textbf\def\PY@tc##1{\textcolor[rgb]{0.50,0.00,0.50}{##1}}}
\@namedef{PY@tok@gd}{\def\PY@tc##1{\textcolor[rgb]{0.63,0.00,0.00}{##1}}}
\@namedef{PY@tok@gi}{\def\PY@tc##1{\textcolor[rgb]{0.00,0.63,0.00}{##1}}}
\@namedef{PY@tok@gr}{\def\PY@tc##1{\textcolor[rgb]{1.00,0.00,0.00}{##1}}}
\@namedef{PY@tok@ge}{\let\PY@it=\textit}
\@namedef{PY@tok@gs}{\let\PY@bf=\textbf}
\@namedef{PY@tok@gp}{\let\PY@bf=\textbf\def\PY@tc##1{\textcolor[rgb]{0.00,0.00,0.50}{##1}}}
\@namedef{PY@tok@go}{\def\PY@tc##1{\textcolor[rgb]{0.53,0.53,0.53}{##1}}}
\@namedef{PY@tok@gt}{\def\PY@tc##1{\textcolor[rgb]{0.00,0.27,0.87}{##1}}}
\@namedef{PY@tok@err}{\def\PY@bc##1{{\setlength{\fboxsep}{\string -\fboxrule}\fcolorbox[rgb]{1.00,0.00,0.00}{1,1,1}{\strut ##1}}}}
\@namedef{PY@tok@kc}{\let\PY@bf=\textbf\def\PY@tc##1{\textcolor[rgb]{0.00,0.50,0.00}{##1}}}
\@namedef{PY@tok@kd}{\let\PY@bf=\textbf\def\PY@tc##1{\textcolor[rgb]{0.00,0.50,0.00}{##1}}}
\@namedef{PY@tok@kn}{\let\PY@bf=\textbf\def\PY@tc##1{\textcolor[rgb]{0.00,0.50,0.00}{##1}}}
\@namedef{PY@tok@kr}{\let\PY@bf=\textbf\def\PY@tc##1{\textcolor[rgb]{0.00,0.50,0.00}{##1}}}
\@namedef{PY@tok@bp}{\def\PY@tc##1{\textcolor[rgb]{0.00,0.50,0.00}{##1}}}
\@namedef{PY@tok@fm}{\def\PY@tc##1{\textcolor[rgb]{0.00,0.00,1.00}{##1}}}
\@namedef{PY@tok@vc}{\def\PY@tc##1{\textcolor[rgb]{0.10,0.09,0.49}{##1}}}
\@namedef{PY@tok@vg}{\def\PY@tc##1{\textcolor[rgb]{0.10,0.09,0.49}{##1}}}
\@namedef{PY@tok@vi}{\def\PY@tc##1{\textcolor[rgb]{0.10,0.09,0.49}{##1}}}
\@namedef{PY@tok@vm}{\def\PY@tc##1{\textcolor[rgb]{0.10,0.09,0.49}{##1}}}
\@namedef{PY@tok@sa}{\def\PY@tc##1{\textcolor[rgb]{0.73,0.13,0.13}{##1}}}
\@namedef{PY@tok@sb}{\def\PY@tc##1{\textcolor[rgb]{0.73,0.13,0.13}{##1}}}
\@namedef{PY@tok@sc}{\def\PY@tc##1{\textcolor[rgb]{0.73,0.13,0.13}{##1}}}
\@namedef{PY@tok@dl}{\def\PY@tc##1{\textcolor[rgb]{0.73,0.13,0.13}{##1}}}
\@namedef{PY@tok@s2}{\def\PY@tc##1{\textcolor[rgb]{0.73,0.13,0.13}{##1}}}
\@namedef{PY@tok@sh}{\def\PY@tc##1{\textcolor[rgb]{0.73,0.13,0.13}{##1}}}
\@namedef{PY@tok@s1}{\def\PY@tc##1{\textcolor[rgb]{0.73,0.13,0.13}{##1}}}
\@namedef{PY@tok@mb}{\def\PY@tc##1{\textcolor[rgb]{0.40,0.40,0.40}{##1}}}
\@namedef{PY@tok@mf}{\def\PY@tc##1{\textcolor[rgb]{0.40,0.40,0.40}{##1}}}
\@namedef{PY@tok@mh}{\def\PY@tc##1{\textcolor[rgb]{0.40,0.40,0.40}{##1}}}
\@namedef{PY@tok@mi}{\def\PY@tc##1{\textcolor[rgb]{0.40,0.40,0.40}{##1}}}
\@namedef{PY@tok@il}{\def\PY@tc##1{\textcolor[rgb]{0.40,0.40,0.40}{##1}}}
\@namedef{PY@tok@mo}{\def\PY@tc##1{\textcolor[rgb]{0.40,0.40,0.40}{##1}}}
\@namedef{PY@tok@ch}{\let\PY@it=\textit\def\PY@tc##1{\textcolor[rgb]{0.25,0.50,0.50}{##1}}}
\@namedef{PY@tok@cm}{\let\PY@it=\textit\def\PY@tc##1{\textcolor[rgb]{0.25,0.50,0.50}{##1}}}
\@namedef{PY@tok@cpf}{\let\PY@it=\textit\def\PY@tc##1{\textcolor[rgb]{0.25,0.50,0.50}{##1}}}
\@namedef{PY@tok@c1}{\let\PY@it=\textit\def\PY@tc##1{\textcolor[rgb]{0.25,0.50,0.50}{##1}}}
\@namedef{PY@tok@cs}{\let\PY@it=\textit\def\PY@tc##1{\textcolor[rgb]{0.25,0.50,0.50}{##1}}}

\def\PYZbs{\char`\\}
\def\PYZus{\char`\_}
\def\PYZob{\char`\{}
\def\PYZcb{\char`\}}
\def\PYZca{\char`\^}
\def\PYZam{\char`\&}
\def\PYZlt{\char`\<}
\def\PYZgt{\char`\>}
\def\PYZsh{\char`\#}
\def\PYZpc{\char`\%}
\def\PYZdl{\char`\$}
\def\PYZhy{\char`\-}
\def\PYZsq{\char`\'}
\def\PYZdq{\char`\"}
\def\PYZti{\char`\~}
% for compatibility with earlier versions
\def\PYZat{@}
\def\PYZlb{[}
\def\PYZrb{]}
\makeatother


    % For linebreaks inside Verbatim environment from package fancyvrb. 
    \makeatletter
        \newbox\Wrappedcontinuationbox 
        \newbox\Wrappedvisiblespacebox 
        \newcommand*\Wrappedvisiblespace {\textcolor{red}{\textvisiblespace}} 
        \newcommand*\Wrappedcontinuationsymbol {\textcolor{red}{\llap{\tiny$\m@th\hookrightarrow$}}} 
        \newcommand*\Wrappedcontinuationindent {3ex } 
        \newcommand*\Wrappedafterbreak {\kern\Wrappedcontinuationindent\copy\Wrappedcontinuationbox} 
        % Take advantage of the already applied Pygments mark-up to insert 
        % potential linebreaks for TeX processing. 
        %        {, <, #, %, $, ' and ": go to next line. 
        %        _, }, ^, &, >, - and ~: stay at end of broken line. 
        % Use of \textquotesingle for straight quote. 
        \newcommand*\Wrappedbreaksatspecials {% 
            \def\PYGZus{\discretionary{\char`\_}{\Wrappedafterbreak}{\char`\_}}% 
            \def\PYGZob{\discretionary{}{\Wrappedafterbreak\char`\{}{\char`\{}}% 
            \def\PYGZcb{\discretionary{\char`\}}{\Wrappedafterbreak}{\char`\}}}% 
            \def\PYGZca{\discretionary{\char`\^}{\Wrappedafterbreak}{\char`\^}}% 
            \def\PYGZam{\discretionary{\char`\&}{\Wrappedafterbreak}{\char`\&}}% 
            \def\PYGZlt{\discretionary{}{\Wrappedafterbreak\char`\<}{\char`\<}}% 
            \def\PYGZgt{\discretionary{\char`\>}{\Wrappedafterbreak}{\char`\>}}% 
            \def\PYGZsh{\discretionary{}{\Wrappedafterbreak\char`\#}{\char`\#}}% 
            \def\PYGZpc{\discretionary{}{\Wrappedafterbreak\char`\%}{\char`\%}}% 
            \def\PYGZdl{\discretionary{}{\Wrappedafterbreak\char`\$}{\char`\$}}% 
            \def\PYGZhy{\discretionary{\char`\-}{\Wrappedafterbreak}{\char`\-}}% 
            \def\PYGZsq{\discretionary{}{\Wrappedafterbreak\textquotesingle}{\textquotesingle}}% 
            \def\PYGZdq{\discretionary{}{\Wrappedafterbreak\char`\"}{\char`\"}}% 
            \def\PYGZti{\discretionary{\char`\~}{\Wrappedafterbreak}{\char`\~}}% 
        } 
        % Some characters . , ; ? ! / are not pygmentized. 
        % This macro makes them "active" and they will insert potential linebreaks 
        \newcommand*\Wrappedbreaksatpunct {% 
            \lccode`\~`\.\lowercase{\def~}{\discretionary{\hbox{\char`\.}}{\Wrappedafterbreak}{\hbox{\char`\.}}}% 
            \lccode`\~`\,\lowercase{\def~}{\discretionary{\hbox{\char`\,}}{\Wrappedafterbreak}{\hbox{\char`\,}}}% 
            \lccode`\~`\;\lowercase{\def~}{\discretionary{\hbox{\char`\;}}{\Wrappedafterbreak}{\hbox{\char`\;}}}% 
            \lccode`\~`\:\lowercase{\def~}{\discretionary{\hbox{\char`\:}}{\Wrappedafterbreak}{\hbox{\char`\:}}}% 
            \lccode`\~`\?\lowercase{\def~}{\discretionary{\hbox{\char`\?}}{\Wrappedafterbreak}{\hbox{\char`\?}}}% 
            \lccode`\~`\!\lowercase{\def~}{\discretionary{\hbox{\char`\!}}{\Wrappedafterbreak}{\hbox{\char`\!}}}% 
            \lccode`\~`\/\lowercase{\def~}{\discretionary{\hbox{\char`\/}}{\Wrappedafterbreak}{\hbox{\char`\/}}}% 
            \catcode`\.\active
            \catcode`\,\active 
            \catcode`\;\active
            \catcode`\:\active
            \catcode`\?\active
            \catcode`\!\active
            \catcode`\/\active 
            \lccode`\~`\~ 	
        }
    \makeatother

    \let\OriginalVerbatim=\Verbatim
    \makeatletter
    \renewcommand{\Verbatim}[1][1]{%
        %\parskip\z@skip
        \sbox\Wrappedcontinuationbox {\Wrappedcontinuationsymbol}%
        \sbox\Wrappedvisiblespacebox {\FV@SetupFont\Wrappedvisiblespace}%
        \def\FancyVerbFormatLine ##1{\hsize\linewidth
            \vtop{\raggedright\hyphenpenalty\z@\exhyphenpenalty\z@
                \doublehyphendemerits\z@\finalhyphendemerits\z@
                \strut ##1\strut}%
        }%
        % If the linebreak is at a space, the latter will be displayed as visible
        % space at end of first line, and a continuation symbol starts next line.
        % Stretch/shrink are however usually zero for typewriter font.
        \def\FV@Space {%
            \nobreak\hskip\z@ plus\fontdimen3\font minus\fontdimen4\font
            \discretionary{\copy\Wrappedvisiblespacebox}{\Wrappedafterbreak}
            {\kern\fontdimen2\font}%
        }%
        
        % Allow breaks at special characters using \PYG... macros.
        \Wrappedbreaksatspecials
        % Breaks at punctuation characters . , ; ? ! and / need catcode=\active 	
        \OriginalVerbatim[#1,codes*=\Wrappedbreaksatpunct]%
    }
    \makeatother

    % Exact colors from NB
    \definecolor{incolor}{HTML}{303F9F}
    \definecolor{outcolor}{HTML}{D84315}
    \definecolor{cellborder}{HTML}{CFCFCF}
    \definecolor{cellbackground}{HTML}{F7F7F7}
    
    % prompt
    \makeatletter
    \newcommand{\boxspacing}{\kern\kvtcb@left@rule\kern\kvtcb@boxsep}
    \makeatother
    \newcommand{\prompt}[4]{
        {\ttfamily\llap{{\color{#2}[#3]:\hspace{3pt}#4}}\vspace{-\baselineskip}}
    }
    

    
    % Prevent overflowing lines due to hard-to-break entities
    \sloppy 
    % Setup hyperref package
    \hypersetup{
      breaklinks=true,  % so long urls are correctly broken across lines
      colorlinks=true,
      urlcolor=urlcolor,
      linkcolor=linkcolor,
      citecolor=citecolor,
      }
    % Slightly bigger margins than the latex defaults
    
    \geometry{verbose,tmargin=1in,bmargin=1in,lmargin=1in,rmargin=1in}
    
    

\begin{document}
    
    \maketitle
    
    

    
    \section{Электрическое и магнитное поля в соответсвии с потенциалом
Лиенара
Вихерта}\label{ux44dux43bux435ux43aux442ux440ux438ux447ux435ux441ux43aux43eux435-ux438-ux43cux430ux433ux43dux438ux442ux43dux43eux435-ux43fux43eux43bux44f-ux432-ux441ux43eux43eux442ux432ux435ux442ux441ux432ux438ux438-ux441-ux43fux43eux442ux435ux43dux446ux438ux430ux43bux43eux43c-ux43bux438ux435ux43dux430ux440ux430-ux432ux438ux445ux435ux440ux442ux430}

    А.Дроздов

    Рассмотрим модель движущихся зарядов и рассчитаем поля движущихся по
заданным траекториям зарядов в соответствии с потенциалами
Лиенара-Вихерта с учётом запаздывания

    Для решения этой задачи выразим электрическое поле исходя из выражения
потенциалов Лиенара Вихерта \cite{LL2}, как известно, дифференцированием
скалярного потенциала ЛВ по координатам точки наблюдения
\(\overrightarrow{E_1} = - \nabla\varphi\) и дифференцированием
векторного потенциала по времени
\(\overrightarrow{E_2}=-\frac{1}{c}\frac{\partial \overrightarrow{A}}{\partial t}\),
где \(\varphi=\frac{q}{{{R}^{*}}}\) и
\(\overrightarrow{A}=\frac{\overrightarrow{v}}{c}\frac{q}{{{R}^{*}}}\)
где
\({{R}^{*}}=\left( R-\frac{\overrightarrow{v}\cdot \overrightarrow{R}}{c} \right)\)
- радиус Лиенара Вихерта.

    радиус вектор из запаздывающей координаты заряда \((x', y', z')\) в
запаздывающий момент \(t'\) в точку наблюдения \((x, y, z)\) в момент
наблюдения \(t\)

\(\vec R = \vec R\left(t'\right)  = \vec i \left(x-x'\left(t'\right)\right)  + \vec j \left(y-y'\left(t'\right)\right)  + \vec k \left(z-z'\left(t'\right)\right)\)

расстояние от точки наблюдения \((x, y, z)\) в момент наблюдения \(t\) к
запаздывающей координате заряда \((x', y', z')\) в запаздывающий момент
\(t'\)

\(R = R\left(t'\right) = c \left(t-t'\right)\)

скорость заряда в запаздывающий момент времени \(t'\)

\(\vec v = \vec v\left(t'\right) = \frac {\partial}{\partial t'}(x', y', z')\)

    Дифференцируем \(R\left(t'\right) = c \left(t-t'\right)\) по \(t\)

\(\frac{\partial R\left(t'\right)}{\partial t} = \frac{\partial R\left(t'\right)}{\partial t'} \frac{\partial t'}{\partial t}\)

    Дифференцируем тождество
\(R\left(t'\right)^2 = \left(\vec R\left(t'\right)^2\right)\) по \(t'\)

\(\frac{\partial}{\partial t'}\left(R\left(t'\right)^2\right) = \frac{\partial}{\partial t'} \left(\vec R\left(t'\right)^2\right)\)

    \(\frac{\partial}{\partial t'}R^2 = \frac{\partial}{\partial t'} \left(\left(\vec i \left(x-x'\left(t'\right)\right)  + \vec j \left(y-y'\left(t'\right)\right)  + \vec k \left(z-z'\left(t'\right)\right)\right)^2\right)\)

    \(2\,R\left(t'\right)\frac{\partial R\left(t'\right)}{\partial t'} = 2\,\vec R\left(t'\right)\frac{\partial \vec R\left(t'\right)}{\partial t'}\)

    Учитывая что
\(\frac{\partial \vec R\left(t'\right)}{\partial t'} = -\vec v\left(t'\right)\)
(знак минус связан с тем что \(R\left(t'\right)\) есть радиус вектор из
запаздывающей координаты заряда \((x', y', z')\) в точку наблюдения
\((x, y, z)\) а не наоборот)

    \(R\left(t'\right)\frac{\partial R\left(t'\right)}{\partial t'} = -\vec R\left(t'\right)\vec v\left(t'\right)\)

    \(\frac{\partial R\left(t'\right)}{\partial t'} = -\frac{\vec R\left(t'\right)\vec v\left(t'\right)}{R\left(t'\right)}\)

    Получаем

\(\frac{\partial R\left(t'\right)}{\partial t} = \frac{\partial R\left(t'\right)}{\partial t'} \frac{\partial t'}{\partial t} = -\frac{\vec R\left(t'\right)\vec v\left(t'\right)}{R\left(t'\right)} \frac{\partial t'}{\partial t} = c\left(1-\frac{\partial t'}{\partial t}\right)\)

Отсюда

\[\frac{\partial t'}{\partial t} = \frac{1}{1 -\frac{\vec R\left(t'\right)\vec v\left(t'\right)}{R\left(t'\right)\,c}} = \frac{R\left(t'\right)}{{{R}^{*}}\left(t'\right)}\]

    Дифференцируем \(R\left(t'\right) = c \left(t-t'\right)\) по координатам
наблюдения

\(\frac{\frac{\partial}{\partial x} R\left(t'\right)}{c} = \frac{\partial}{\partial x} \left(t-t'\right)\)

\(\frac{\partial}{\partial x} t' = -\frac{\frac{\partial}{\partial x} R\left(t'\right)}{c} = - \frac{1}{c}\left(\frac{\partial R\left(t'\right)}{\partial t'} \frac{\partial}{\partial x} t' + \frac{R_x}{R}\right)\)

    \(\frac{\partial}{\partial x} t' = - \frac{1}{c}\left(-\frac{\vec R\left(t'\right)\vec v\left(t'\right)}{R\left(t'\right)} \frac{\partial}{\partial x} t' + \frac{R_x}{R}\right)\)

Отсюда

\[\frac{\partial  t'}{\partial x} = - \frac{R_x\left(t'\right)}{c\,\left(R\left(t'\right) -\frac{\vec R\left(t'\right)\vec v\left(t'\right)}{c}\right)} = - \frac{R_x\left(t'\right)}{c\,{{R}^{*}}\left(t'\right)}\]

    Дифференцируем \(R\left(t'\right) = c \left(t-t'\right)\) по координатам
наблюдения

\(\frac{\nabla R\left(t'\right)}{c} = \nabla \left(t-t'\right)\)

\(\nabla t' = -\frac{\nabla R\left(t'\right)}{c} = - \frac{1}{c}\left(\frac{\partial R\left(t'\right)}{\partial t'} \nabla t' + \frac{\vec R}{R}\right)\)

    \(\nabla t' = - \frac{1}{c}\left(-\frac{\vec R\left(t'\right)\vec v\left(t'\right)}{R\left(t'\right)} \nabla t' + \frac{\vec R}{R}\right)\)

Отсюда

\[\nabla t' = - \frac{\vec R\left(t'\right)}{c\,\left(R\left(t'\right) -\frac{\vec R\left(t'\right)\vec v\left(t'\right)}{c}\right)}  = - \frac{\vec R\left(t'\right)}{c\,{{R}^{*}}\left(t'\right)}\]

    С помощью этих формул вычисляем поля

электрическое поле:

\(\vec{E} = -\nabla\varphi - \frac{1}{c}\frac{\partial}{\partial t}\vec{A}\)

где

\(\varphi=\frac{q}{{R}^{*}}\),
\(\vec{A}=\frac{\vec{v}}{c}\frac{q}{{R}^{*}}\)

и

\({R}^{*} = R - \frac{\vec{v}\vec{R}}{c}\)

Итак, поскольку

\(\nabla\frac{1}{{{R}^{*}}} = \left(\frac{d}{d{{R}^{*}}}\frac{1}{{{R}^{*}}}\right)\left(\nabla {{R}^{*}}\right) = -\frac{1}{{{R}^{*}}^2}\nabla {{R}^{*}}\)

    Нам теперь нужен градиент по координатам наблюдения радиуса Лиенара
Вихерта

\(\nabla {{R}^{*}} = \nabla {\left( R-\frac{\overrightarrow{v}\cdot \overrightarrow{R}}{c} \right)} = {\left( \nabla R-\nabla \frac{\overrightarrow{v}\cdot \overrightarrow{R}}{c} \right)}\)

    находим градиент расстояния

\(\nabla R = -\,с \,\nabla t' = \,c\,\frac{\vec R}{c\,\left(R -\frac{\vec R\vec v}{c}\right)} = \frac{\vec R}{R -\frac{\vec R\vec v}{c}}=\frac{\vec R}{{R}^{*}}\)

    Далее учитывая формулу для градиента скалярного произведения

\(\nabla \left(\overrightarrow{v}\cdot \overrightarrow{R}\right)  = \overrightarrow{v} \times \left(\nabla \times \overrightarrow{R}\right)  + \overrightarrow{R} \times \left(\nabla \times \overrightarrow{v}\right)  + \left(\overrightarrow{v} \cdot \nabla\right) \overrightarrow{R}  + \left(\overrightarrow{R} \cdot \nabla\right) \overrightarrow{v}\)

Для первого слагаемого находим

    \(\left(\nabla \times \overrightarrow{R}\right) = rot \overrightarrow{R}\)

    \(rot \overrightarrow{R}\left(t'\right)=\left|\begin{array}{ccc}  \vec i & \vec j & \vec k\\  \frac{\partial}{\partial x} & \frac{\partial}{\partial y} & \frac{\partial}{\partial z}\\  R_{x}\left(t'\right) & R_{y}\left(t'\right) & R_{z}\left(t'\right)\\ \end{array}\right|\)

    \(rot \overrightarrow{R}\left(t'\right)=\left|\begin{array}{ccc}  \vec i & \vec j & \vec k\\  \frac{\partial}{\partial t'}\frac{\partial t'}{\partial x} & \frac{\partial}{\partial t'}\frac{\partial t'}{\partial y} & \frac{\partial}{\partial t'}\frac{\partial t'}{\partial z}\\  R_{x}\left(t'\right) & R_{y}\left(t'\right) & R_{z}\left(t'\right)\\ \end{array}\right|\)

    \(rot \overrightarrow{R}\left(t'\right)=\left|\begin{array}{ccc}  \vec i & \vec j & \vec k \\  -\frac{R_x\left(t'\right)}{c\,{{R}^{*}}\left(t'\right)}\frac{\partial}{\partial t'}  & -\frac{R_y\left(t'\right)}{c\,{{R}^{*}}\left(t'\right)}\frac{\partial}{\partial t'}  & -\frac{R_z\left(t'\right)}{c\,{{R}^{*}}\left(t'\right)}\frac{\partial}{\partial t'} \\  R_{x}\left(t'\right) & R_{y}\left(t'\right) & R_{z}\left(t'\right)\\ \end{array}\right|\)

    \(rot \overrightarrow{R}\left(t'\right)=-\frac{1}{c\,{{R}^{*}}\left(t'\right)}\left|\begin{array}{ccc}  \vec i & \vec j & \vec k \\  R_x\left(t'\right)\frac{\partial}{\partial t'}  & R_y\left(t'\right)\frac{\partial}{\partial t'}  & R_z\left(t'\right)\frac{\partial}{\partial t'} \\  R_{x}\left(t'\right) & R_{y}\left(t'\right) & R_{z}\left(t'\right)\\ \end{array}\right|\)

    \(rot \overrightarrow{R}\left(t'\right)=-\frac{1}{c\,{{R}^{*}}\left(t'\right)}\left|\begin{array}{ccc}  \vec i & \vec j & \vec k \\  R_x\left(t'\right)  & R_y\left(t'\right)  & R_z\left(t'\right) \\  \frac{\partial}{\partial t'}R_{x}\left(t'\right)  & \frac{\partial}{\partial t'} R_{y}\left(t'\right)  & \frac{\partial}{\partial t'} R_{z}\left(t'\right)\\ \end{array}\right|\)

    \(rot \overrightarrow{R}\left(t'\right)=-\frac{1}{c\,{{R}^{*}}\left(t'\right)}\left|\begin{array}{ccc}  \vec i & \vec j & \vec k \\  R_x\left(t'\right)  & R_y\left(t'\right)  & R_z\left(t'\right) \\  -v_{x}\left(t'\right)  & -v_{y}\left(t'\right)  & -v_{z}\left(t'\right)\\ \end{array}\right|\)

    \(rot \overrightarrow{R}=\frac{1}{c\,{{R}^{*}}}\left(  \vec i \left(v_z R_y - v_y R_z\right) +  \vec j \left(v_x R_z - v_z R_x\right) +  \vec k \left(v_y R_x - v_x R_y\right)\right)\)

    Далее

\(\vec v \times rot \overrightarrow{R}=\frac{1}{c\,{{R}^{*}}}\left|\begin{array}{ccc}  \vec i & \vec j & \vec k \\  v_x  & v_y  & v_z \\  \left(v_z R_y - v_y R_z\right)  & \left(v_x R_z - v_z R_x\right)  & \left(v_y R_x - v_x R_y\right)\\ \end{array}\right|\)

    \(\vec v \times rot \overrightarrow{R}=\frac{1}{c\,{{R}^{*}}}\left(\begin{array}{c}  +\vec i \left( v_y \left(v_y R_x - v_x R_y\right) - v_z \left(v_x R_z - v_z R_x\right) \right) \\  +\vec j \left( v_z \left(v_z R_y - v_y R_z\right) - v_x \left(v_y R_x - v_x R_y\right) \right) \\  +\vec k \left( v_x \left(v_x R_z - v_z R_x\right) - v_y \left(v_z R_y - v_y R_z\right) \right) \end{array}\right)\)

    \(\vec v \times rot \overrightarrow{R}=\frac{1}{c\,{{R}^{*}}}\left(\begin{array}{c}  +\vec i \left(R_x\left( v_y^2 + v_z^2\right) - v_x v_y R_y - v_x v_z R_z \right) \\  +\vec i \left(R_y\left( v_z^2 + v_x^2\right) - v_y v_z R_z - v_y v_x R_x \right) \\  +\vec i \left(R_z\left( v_x^2 + v_y^2\right) - v_z v_x R_x - v_z v_y R_y \right) \\ \end{array}\right)\)

    \(\vec v \times rot \overrightarrow{R}=\frac{1}{c\,{{R}^{*}}}\left(\begin{array}{c}  +\vec i \left(R_x\left( v_y^2 + v_z^2\right) - v_x \left(v_y R_y + v_z R_z \right) \right) \\  +\vec i \left(R_y\left( v_z^2 + v_x^2\right) - v_y \left(v_z R_z + v_x R_x \right) \right) \\  +\vec i \left(R_z\left( v_x^2 + v_y^2\right) - v_z \left(v_x R_x + v_y R_y \right) \right) \\ \end{array}\right)\)

    \(\vec v \times rot \overrightarrow{R}=\frac{1}{c\,{{R}^{*}}}\left(\begin{array}{c}  +\vec i \left(R_x\left(v_x^2 + v_y^2 + v_z^2\right) - v_x \left(v_x R_x + v_y R_y + v_z R_z \right) \right) \\  +\vec i \left(R_y\left(v_y^2 + v_z^2 + v_x^2\right) - v_y \left(v_y R_y + v_z R_z + v_x R_x \right) \right) \\  +\vec i \left(R_z\left(v_z^2 + v_x^2 + v_y^2\right) - v_z \left(v_z R_z + v_x R_x + v_y R_y \right) \right) \\ \end{array}\right)\)

    Переходя к векторным обозначениям

\(\vec v \times rot \overrightarrow{R}=\frac{1}{c\,{{R}^{*}}}\left(v^2 \vec R - \left(\vec v \vec R\right) \vec v\right)\)

    Для второго слагаемого находим

\(\left(\nabla \times \overrightarrow{v}\right) = rot \overrightarrow{v}\)

    \(rot \overrightarrow{v}\left(t'\right)=\left|\begin{array}{ccc}  \vec i & \vec j & \vec k\\  \frac{\partial}{\partial x} & \frac{\partial}{\partial y} & \frac{\partial}{\partial z}\\  v_{x}\left(t'\right) & v_{y}\left(t'\right) & v_{z}\left(t'\right)\\ \end{array}\right|\)

    \(rot \overrightarrow{v}\left(t'\right)=\left|\begin{array}{ccc}  \vec i & \vec j & \vec k\\  \frac{\partial}{\partial t'}\frac{\partial t'}{\partial x} & \frac{\partial}{\partial t'}\frac{\partial t'}{\partial y} & \frac{\partial}{\partial t'}\frac{\partial t'}{\partial z}\\  v_{x}\left(t'\right) & v_{y}\left(t'\right) & v_{z}\left(t'\right)\\ \end{array}\right|\)

    \(rot \overrightarrow{v}\left(t'\right)=\left|\begin{array}{ccc}  \vec i & \vec j & \vec k \\  -\frac{R_x\left(t'\right)}{c\,{{R}^{*}}\left(t'\right)}\frac{\partial}{\partial t'}  & -\frac{R_y\left(t'\right)}{c\,{{R}^{*}}\left(t'\right)}\frac{\partial}{\partial t'}  & -\frac{R_z\left(t'\right)}{c\,{{R}^{*}}\left(t'\right)}\frac{\partial}{\partial t'} \\  v_{x}\left(t'\right) & v_{y}\left(t'\right) & v_{z}\left(t'\right)\\ \end{array}\right|\)

    \(rot \overrightarrow{v}\left(t'\right)=-\frac{1}{c\,{{R}^{*}}\left(t'\right)}\left|\begin{array}{ccc}  \vec i & \vec j & \vec k \\  R_x\left(t'\right)\frac{\partial}{\partial t'}  & R_y\left(t'\right)\frac{\partial}{\partial t'}  & R_z\left(t'\right)\frac{\partial}{\partial t'} \\  v_{x}\left(t'\right) & v_{y}\left(t'\right) & v_{z}\left(t'\right)\\ \end{array}\right|\)

    \(rot \overrightarrow{v}\left(t'\right)=-\frac{1}{c\,{{R}^{*}}\left(t'\right)}\left|\begin{array}{ccc}  \vec i & \vec j & \vec k \\  R_x\left(t'\right)  & R_y\left(t'\right)  & R_z\left(t'\right) \\  \frac{\partial}{\partial t'}v_{x}\left(t'\right)  & \frac{\partial}{\partial t'} v_{y}\left(t'\right)  & \frac{\partial}{\partial t'} v_{z}\left(t'\right)\\ \end{array}\right|\)

    \(rot \overrightarrow{v}\left(t'\right)=-\frac{1}{c\,{{R}^{*}}\left(t'\right)}\left|\begin{array}{ccc}  \vec i & \vec j & \vec k \\  R_x\left(t'\right)  & R_y\left(t'\right)  & R_z\left(t'\right) \\  a_{x}\left(t'\right)  & a_{y}\left(t'\right)  & a_{z}\left(t'\right)\\ \end{array}\right|\)

    \(rot \overrightarrow{v}=-\frac{1}{c\,{{R}^{*}}}\left(  \vec i \left(a_z R_y - a_y R_z\right) +  \vec j \left(a_x R_z - a_z R_x\right) +  \vec k \left(a_y R_x - a_x R_y\right)\right)\)

    Далее

\(\vec R \times rot \overrightarrow{v}=-\frac{1}{c\,{{R}^{*}}}\left|\begin{array}{ccc}  \vec i & \vec j & \vec k \\  R_x  & R_y  & R_z \\  \left(a_z R_y - a_y R_z\right)  & \left(a_x R_z - a_z R_x\right)  & \left(a_y R_x - a_x R_y\right)\\ \end{array}\right|\)

    \(\vec R \times rot \overrightarrow{v}=-\frac{1}{c\,{{R}^{*}}}\left(\begin{array}{c}  +\vec i \left( R_y \left(a_y R_x - a_x R_y\right) - R_z \left(a_x R_z - a_z R_x\right) \right) \\  +\vec j \left( R_z \left(a_z R_y - a_y R_z\right) - R_x \left(a_y R_x - a_x R_y\right) \right) \\  +\vec k \left( R_x \left(a_x R_z - a_z R_x\right) - R_y \left(a_z R_y - a_y R_z\right) \right) \end{array}\right)\)

    \(\vec R \times rot \overrightarrow{v}=-\frac{1}{c\,{{R}^{*}}}\left(\begin{array}{c}  +\vec i \left(a_x\left( R_y^2 + R_z^2\right) - R_x R_y a_y - R_x R_z a_z \right) \\  +\vec i \left(a_y\left( R_z^2 + R_x^2\right) - R_y R_z a_z - R_y R_x a_x \right) \\  +\vec i \left(a_z\left( R_x^2 + R_y^2\right) - R_z R_x a_x - R_z R_y a_y \right) \\ \end{array}\right)\)

    \(\vec R \times rot \overrightarrow{v}=-\frac{1}{c\,{{R}^{*}}}\left(\begin{array}{c}  +\vec i \left(a_x\left( R_y^2 + R_z^2\right) - R_x \left(R_y a_y + R_z a_z \right) \right) \\  +\vec i \left(a_y\left( R_z^2 + R_x^2\right) - R_y \left(R_z a_z + R_x a_x \right) \right) \\  +\vec i \left(a_z\left( R_x^2 + R_y^2\right) - R_z \left(R_x a_x + R_y a_y \right) \right) \\ \end{array}\right)\)

    \(\vec R \times rot \overrightarrow{v}=-\frac{1}{c\,{{R}^{*}}}\left(\begin{array}{c}  +\vec i \left(a_x\left(R_x^2 + R_y^2 + R_z^2\right) - R_x \left(R_x a_x + R_y a_y + R_z a_z \right) \right) \\  +\vec i \left(a_y\left(R_y^2 + R_z^2 + R_x^2\right) - R_y \left(R_y a_y + R_z a_z + R_x a_x \right) \right) \\  +\vec i \left(a_z\left(R_z^2 + R_x^2 + R_y^2\right) - R_z \left(R_z a_z + R_x a_x + R_y a_y \right) \right) \\ \end{array}\right)\)

    Переходя к векторным обозначениям

\(\vec R \times rot \overrightarrow{v}=-\frac{1}{c\,{{R}^{*}}}\left(R^2 \vec a - \left(\vec R \vec a\right) \vec R\right)\)

    Третье слагаемое

\(\left(\overrightarrow{v}\left(t'\right) \cdot \nabla\right) \overrightarrow{R}\left(t'\right)  = v_x\left(t'\right) \frac{\partial}{\partial x} \overrightarrow{R}\left(t'\right)  + v_y\left(t'\right) \frac{\partial}{\partial y} \overrightarrow{R}\left(t'\right)  + v_z\left(t'\right) \frac{\partial}{\partial z} \overrightarrow{R}\left(t'\right)\)

    \(\frac{\partial}{\partial x} \vec R\left(t'\right)  = \frac{\partial \vec R\left(t'\right)}{\partial t'} \frac{\partial t'}{\partial x} + \frac{\partial \vec R}{\partial x}\Big|_{t'=const}  = - \vec v\left(t'\right) \frac{\partial t'}{\partial x} + \frac{\partial \vec R}{\partial x}\Big|_{t'=const}  = - \vec v\left(t'\right) \frac{R_x\left(t'\right)}{c\,{{R}^{*}}\left(t'\right)} + \frac{\partial \vec R}{\partial x}\Big|_{t'=const}\)

    \(\frac{\partial\vec R\left(t'\right)}{\partial x} \Big|_{t'=const}  = \frac{\partial}{\partial x}\left(\vec i \left(x-x'\left(t'\right)\right)  + \vec j \left(y-y'\left(t'\right)\right)  + \vec k \left(z-z'\left(t'\right)\right)\right)  = \vec i\)

    \(\frac{\partial}{\partial x} \vec R\left(t'\right) = \vec v \frac{R_x}{c\,{{R}^{*}}} + \vec i\)

    \(\frac{\partial}{\partial y} \vec R\left(t'\right) = \vec v \frac{R_y}{c\,{{R}^{*}}} + \vec j\)

    \(\frac{\partial}{\partial z} \vec R\left(t'\right) = \vec v \frac{R_z}{c\,{{R}^{*}}} + \vec k\)

    \(\left(\overrightarrow{v}\left(t'\right) \cdot \nabla\right) \overrightarrow{R}\left(t'\right)  = v_x\left(\vec v \frac{R_x}{c\,{{R}^{*}}} + \vec i\right)  + v_y\left(\vec v \frac{R_y}{c\,{{R}^{*}}} + \vec j\right)  + v_z\left(\vec v \frac{R_z}{c\,{{R}^{*}}} + \vec k\right)  = \vec v + \vec v \frac{\left(\vec R \cdot \vec v\right)}{c\,{{R}^{*}}}\)

    Четвертое слагаемое

\(\left(\overrightarrow{R}\left(t'\right) \cdot \nabla\right) \overrightarrow{v}\left(t'\right)  = R_x\left(t'\right) \frac{\partial}{\partial x} \overrightarrow{v}\left(t'\right)  + R_y\left(t'\right) \frac{\partial}{\partial y} \overrightarrow{v}\left(t'\right)  + R_z\left(t'\right) \frac{\partial}{\partial z} \overrightarrow{v}\left(t'\right)\)

    \(\frac{\partial}{\partial x} \vec v\left(t'\right)  = \frac{\partial \vec v\left(t'\right)}{\partial t'} \frac{\partial t'}{\partial x}  = \vec a\left(t'\right) \frac{\partial t'}{\partial x}  = \frac{\vec a\left(t'\right)}{c} \frac{R_x\left(t'\right)}{{{R}^{*}}\left(t'\right)}\)

    \(\left(\overrightarrow{R}\left(t'\right) \cdot \nabla\right) \overrightarrow{v}\left(t'\right)  = R_x \frac{\vec a}{c} \frac{R_x}{{{R}^{*}}}  + R_y \frac{\vec a}{c} \frac{R_y}{{{R}^{*}}}  + R_z \frac{\vec a}{c} \frac{R_z}{{{R}^{*}}}  = \frac{\vec a}{c} \frac{R^2}{{{R}^{*}}}\)

    \begin{tcolorbox}[breakable, size=fbox, boxrule=1pt, pad at break*=1mm,colback=cellbackground, colframe=cellborder]
\prompt{In}{incolor}{ }{\boxspacing}
\begin{Verbatim}[commandchars=\\\{\}]

\end{Verbatim}
\end{tcolorbox}

    Подводя итоги для

\(\nabla \left(\overrightarrow{v}\cdot \overrightarrow{R}\right)  = \overrightarrow{v} \times \left(\nabla \times \overrightarrow{R}\right)  + \overrightarrow{R} \times \left(\nabla \times \overrightarrow{v}\right)  + \left(\overrightarrow{v} \cdot \nabla\right) \overrightarrow{R}  + \left(\overrightarrow{R} \cdot \nabla\right) \overrightarrow{v}\)

    \(\vec v \times rot \overrightarrow{R}=\frac{1}{c\,{{R}^{*}}}\left(v^2 \vec R - \left(\vec v \vec R\right) \vec v\right)\)

    \(\vec R \times rot \overrightarrow{v}=-\frac{1}{c\,{{R}^{*}}}\left(R^2 \vec a - \left(\vec R \vec a\right) \vec R\right)\)

    \(\left(\overrightarrow{v}\left(t'\right) \cdot \nabla\right) \overrightarrow{R}\left(t'\right)  = \vec v + \vec v \frac{\left(\vec R \cdot \vec v\right)}{c\,{{R}^{*}}}\)

    \(\left(\overrightarrow{R}\left(t'\right) \cdot \nabla\right) \overrightarrow{v}\left(t'\right)  = \frac{\vec a}{c} \frac{R^2}{{{R}^{*}}}\)

    Находим для градиента скалярного произведения скорости на радиус вектор

\(\nabla \left(\overrightarrow{v}\cdot \overrightarrow{R}\right)  = \vec v + \frac{1}{c\,{{R}^{*}}}\left(v^2 \vec R - \left(\vec R \vec a\right) \vec R\right)\)

    Таким образом для градиента по координатам наблюдения радиуса Лиенара
Вихерта

\(\nabla {{R}^{*}}  = \nabla {\left( R-\frac{\overrightarrow{v}\cdot \overrightarrow{R}}{c} \right)}  = {\left( \frac{\vec R}{{R}^{*}}-\nabla \frac{\overrightarrow{v}\cdot \overrightarrow{R}}{c} \right)}  = {\left( \frac{\vec R}{{R}^{*}}- \frac{\vec v}{c} - \frac{1}{c^2\,{{R}^{*}}}\left(v^2 \vec R - \left(\vec R \vec a\right) \vec R\right) \right)}\)

    Градиент по координатам наблюдения радиуса Лиенара Вихерта

\(\nabla {{R}^{*}} = \nabla {\left( R-\frac{\overrightarrow{v}\cdot \overrightarrow{R}}{c} \right)} = \left(\frac{\vec{R}}{{{R}^{*}}}\right)\left(1 + \frac{\vec{R}\vec{a}}{c^2}-\frac{v^2}{c^2}\right)-\frac{\vec{v}}{c}\)

    А для производной по времени наблюдения векторного потенциала

\(\frac{1}{c}\frac{\partial }{\partial t}\overrightarrow{A}  = \frac{q}{c^2}\frac{\partial }{\partial t}\frac{\overrightarrow{v}\left(t'\right)}{{{R}^{*}}}  = \frac{q}{c^2}\frac{\partial }{\partial t}\left(\frac{\overrightarrow{v}\left(t'\right)}{{{R}^{*}}}\right)  = \frac{q}{c^2}\frac{\partial }{\partial t}\left(\overrightarrow{v}\left(t'\right) \cdot \frac{1}{{{R}^{*}}}\right)  = \frac{q}{c^2}\left(\frac{\partial }{\partial t}\overrightarrow{v}\left(t'\right) \cdot \frac{1}{{{R}^{*}}} + \overrightarrow{v}\left(t'\right) \cdot \frac{\partial }{\partial t}\frac{1}{{{R}^{*}}}\right)\)

    \(\frac{\partial }{\partial t}\overrightarrow{v}\left(t'\right) = \frac{\partial \overrightarrow{v}\left(t'\right)}{\partial t'}\frac{\partial t'}{\partial t} = \vec a\left(t'\right) \frac{R\left(t'\right)}{{{R}^{*}}\left(t'\right)}\)

    \(\frac{\partial }{\partial t}\frac{1}{{{R}^{*}}} = \left(\frac{d}{d{{R}^{*}}}\frac{1}{{{R}^{*}}}\right)\left(\frac{\partial }{\partial t} {{R}^{*}}\right) = -\frac{1}{{{R}^{*}}^2}\frac{\partial }{\partial t} {{R}^{*}}\)

    Таким образом для производной по \(t\) радиуса Лиенара Вихерта

\(\frac{\partial }{\partial t} {{R}^{*}}  = \frac{\partial }{\partial t} {\left( R-\frac{\overrightarrow{v}\cdot \overrightarrow{R}}{c} \right)}\)

    Дифференцируем \(R\left(t'\right) = c \left(t-t'\right)\) по \(t\)

\(\frac{\partial R\left(t'\right)}{\partial t} = c \left(1 - \frac{R\left(t'\right)}{{{R}^{*}}\left(t'\right)}\right)\)

    \(\frac{\partial }{\partial t} \vec R\left(t'\right)  = \frac{\partial \vec R\left(t'\right)}{\partial t'}\frac{\partial t'}{\partial t}  = -\vec v\left(t'\right)\frac{R\left(t'\right)}{{{R}^{*}}\left(t'\right)}\)

    \(\frac{\partial }{\partial t}{\left(\overrightarrow{v}\cdot \overrightarrow{R} \right)}  = {\left(\frac{\partial }{\partial t}\overrightarrow{v}\cdot \overrightarrow{R} + \overrightarrow{v}\cdot \frac{\partial }{\partial t}\overrightarrow{R} \right)}  = {\left(\vec a\left(t'\right) \frac{R\left(t'\right)}{{{R}^{*}}\left(t'\right)}\cdot \overrightarrow{R} - \overrightarrow{v}\cdot \vec v\left(t'\right)\frac{R\left(t'\right)}{{{R}^{*}}\left(t'\right)} \right)}  = {\left(\vec a \vec R - v^2 \right)\frac{R}{{{R}^{*}}}}\)

    Таким образом для производной по \(t\) радиуса Лиенара Вихерта

\(\frac{\partial }{\partial t} {{R}^{*}}  = \frac{\partial }{\partial t} {\left( R-\frac{\overrightarrow{v}\cdot \overrightarrow{R}}{c} \right)}  = \left( c \left(1 - \frac{R}{{{R}^{*}}}\right)-\frac{\left(\vec a \vec R - v^2 \right)}{c}\frac{R}{{{R}^{*}}} \right)\)

    А для производной по времени наблюдения векторного потенциала

\(\begin{array}{l}\frac{1}{c}\frac{\partial }{\partial t}\overrightarrow{A} \\  = \frac{q}{c^2}\left(\frac{\partial }{\partial t}\overrightarrow{v}\left(t'\right) \cdot \frac{1}{{{R}^{*}}} + \overrightarrow{v}\left(t'\right) \cdot \frac{\partial }{\partial t}\frac{1}{{{R}^{*}}}\right) \\  = \frac{q}{c^2}\left(\frac{\partial }{\partial t}\overrightarrow{v}\left(t'\right) \cdot \frac{1}{{{R}^{*}}} - \overrightarrow{v}\left(t'\right) \cdot \frac{1}{{{R}^{*}}^2}\frac{\partial }{\partial t} {{R}^{*}}\right) \\  = \frac{q}{c^2}\left(\vec a\left(t'\right) \frac{R\left(t'\right)}{{{R}^{*}}\left(t'\right)} \cdot \frac{1}{{{R}^{*}}} - \overrightarrow{v}\left(t'\right) \cdot \frac{1}{{{R}^{*}}^2}\frac{\partial }{\partial t} {{R}^{*}}\right) \\  = \frac{q}{c^2}\left(\vec a\left(t'\right) \frac{R\left(t'\right)}{{{R}^{*}}\left(t'\right)} \cdot \frac{1}{{{R}^{*}}} - \overrightarrow{v}\left(t'\right) \cdot \frac{1}{{{R}^{*}}^2}\left( c \left(1 - \frac{R}{{{R}^{*}}}\right)-\frac{\left(\vec a \vec R - v^2 \right)}{c}\frac{R}{{{R}^{*}}} \right)\right) \\  = \frac{q}{c^2}\left(\vec a \frac{R}{{{R}^{*}}} \cdot \frac{1}{{{R}^{*}}} - \overrightarrow{v} \cdot \frac{1}{{{R}^{*}}^2}\left( c \left(1 - \frac{R}{{{R}^{*}}}\right)-\frac{\left(\vec a \vec R - v^2 \right)}{c}\frac{R}{{{R}^{*}}} \right)\right) \\  = \frac{q}{{{R}^{*}}^2}\left(\vec a \frac{R}{c^2} - \overrightarrow{v} \cdot \frac{1}{c^2}\left( c \left(1 - \frac{R}{{{R}^{*}}}\right)-\frac{\left(\vec a \vec R - v^2 \right)}{c}\frac{R}{{{R}^{*}}} \right)\right) \\  = \frac{q}{{{R}^{*}}^2}\left(\vec a \frac{R}{c^2} - \frac{\overrightarrow{v}}{c}\left( \left(1 - \frac{R}{{{R}^{*}}}\right)-\frac{\left(\vec a \vec R - v^2 \right)}{c^2}\frac{R}{{{R}^{*}}} \right)\right) \\  = \frac{q}{{{R}^{*}}^2}\left(\vec a \frac{R}{c^2} - \frac{\overrightarrow{v}}{c}\left( 1 - \frac{R}{{{R}^{*}}}-\left(\frac{\vec a \vec R}{c^2} - \frac{v^2}{c^2} \right)\frac{R}{{{R}^{*}}} \right)\right) \\  = \frac{q}{{{R}^{*}}^2}\left(\vec a \frac{R}{c^2} - \frac{\overrightarrow{v}}{c}\left( 1 - \frac{R}{{{R}^{*}}}+\left(\frac{v^2}{c^2} -\frac{\vec a \vec R}{c^2}\right)\frac{R}{{{R}^{*}}} \right)\right) \\  = \frac{q}{{{R}^{*}}^2}\left(\vec a \frac{R}{c^2} - \frac{\overrightarrow{v}}{c}\left( 1 +\left(\frac{v^2}{c^2} -\frac{\vec a \vec R}{c^2} - 1\right)\frac{R}{{{R}^{*}}} \right)\right) \\  = \frac{q}{{{R}^{*}}^2}\left(\vec a \frac{R}{c^2} - \left( \frac{\overrightarrow{v}}{c} +\frac{\overrightarrow{v}}{c}\frac{R}{{{R}^{*}}}\left(\frac{v^2}{c^2} -\frac{\vec a \vec R}{c^2} - 1\right) \right)\right) \\  \end{array}\)

    электрическое поле:

\(\vec{E} = -\nabla\varphi - \frac{1}{c}\frac{\partial}{\partial t}\vec{A}\)

    Градиент по координатам наблюдения радиуса Лиенара Вихерта

\(\nabla {{R}^{*}} = \nabla {\left( R-\frac{\overrightarrow{v}\cdot \overrightarrow{R}}{c} \right)} = \left(\frac{\vec{R}}{{{R}^{*}}}\right)\left(1 + \frac{\vec{R}\vec{a}}{c^2}-\frac{v^2}{c^2}\right)-\frac{\vec{v}}{c}\)

    Учитывая вышеприведенные выкладки, для градиента по координатам
наблюдения скалярного потенциала ЛВ можно привести

\(-\nabla \varphi =-\nabla \frac{q}{{{R}^{*}}}=\frac{q}{{{R}^{*}}^{2}}\left\{ \frac{\overrightarrow{R}}{{{R}^{*}}}\left( 1+\frac{\overrightarrow{a}\cdot \overrightarrow{R}}{{{c}^{2}}}-\frac{\overrightarrow{v}\cdot \overrightarrow{v}}{{{c}^{2}}} \right)-\frac{\overrightarrow{v}}{c} \right\}\)

А для производной по времени наблюдения векторного потенциала

\(-\frac{1}{c}\frac{\partial }{\partial t}\overrightarrow{A}=\frac{q}{{{R}^{*}}^{2}}\left\{ -\frac{\overrightarrow{a}R}{{{c}^{2}}}+\frac{\overrightarrow{v}}{c}\left( \frac{R}{{{R}^{*}}}\left( -1-\frac{\overrightarrow{a}\cdot \overrightarrow{R}}{{{c}^{2}}}+\frac{\overrightarrow{v}\cdot \overrightarrow{v}}{{{c}^{2}}} \right)+1 \right) \right\}\)

\(-\frac{1}{c}\frac{\partial }{\partial t}\overrightarrow{A}=\frac{q}{{{R}^{*}}^{2}}\left\{ -\frac{\overrightarrow{a}R}{{{c}^{2}}}+\frac{\overrightarrow{v}}{c}\left( -\frac{R}{{{R}^{*}}}\left( 1+\frac{\overrightarrow{a}\cdot \overrightarrow{R}}{{{c}^{2}}}-\frac{v^2}{{{c}^{2}}} \right)+1 \right) \right\}\)

    \(-\frac{1}{c}\frac{\partial }{\partial t}\overrightarrow{A}=\frac{q}{{{R}^{*}}^{2}}\left\{ -\frac{\overrightarrow{a}R}{{{c}^{2}}}-\frac{R}{{{R}^{*}}}\frac{\overrightarrow{v}}{c}\left( 1+\frac{\overrightarrow{a}\cdot \overrightarrow{R}}{{{c}^{2}}}-\frac{v^2}{{{c}^{2}}} \right)+\frac{\overrightarrow{v}}{c} \right\}\)

Суммарное электрическое поле

\[\vec{E} = \frac{q}{{{R}^{*}}^{3}}\left( \left(\vec{R}-\frac{R}{c}\vec{v} \right) \left(1 + \frac{\vec{R}\vec{a}}{c^2} - \frac{v^2}{c^2} \right) - \vec{a}\frac{{R}^{*}R}{c^2} \right)\]

    Опуская громоздкие промежуточные выкладки \cite{rustot}, для градиента
по координатам наблюдения скалярного потенциала ЛВ можно привести
\[\nabla \varphi =\nabla \frac{dq}{{{R}^{*}}}=-\frac{dq}{{{R}^{*}}^{2}}\left\{ \frac{\overrightarrow{R}}{{{R}^{*}}}\left( 1+\frac{\overrightarrow{a}\cdot \overrightarrow{R}}{{{c}^{2}}}-\frac{\overrightarrow{v}\cdot \overrightarrow{v}}{{{c}^{2}}} \right)-\frac{\overrightarrow{v}}{c} \right\}\]

А для производной по времени наблюдения векторного потенциала
\[\frac{1}{c}\frac{\partial }{\partial t}\overrightarrow{A}=\frac{1}{c}\frac{\partial }{\partial t}\frac{dq\overrightarrow{v}}{c{{R}^{*}}}=\frac{dq}{{{R}^{*}}^{2}}\left\{ \frac{\overrightarrow{a}R}{{{c}^{2}}}-\frac{\overrightarrow{v}}{c}\left( \frac{R}{{{R}^{*}}}\left( -1-\frac{\overrightarrow{a}\cdot \overrightarrow{R}}{{{c}^{2}}}+\frac{\overrightarrow{v}\cdot \overrightarrow{v}}{{{c}^{2}}} \right)+1 \right) \right\}\]

    Сравнение с выкладками Рустота

http://www.sciteclibrary.ru/cgi-bin/yabb2/YaBB.pl?num=1528093569/326\#326

\(\varphi=\frac{q}{{R}^{*}}\),
\(\vec{A}=\frac{\vec{v}}{c}\frac{q}{{R}^{*}}\)

\({R}^{*} = R - \frac{\vec{v}\vec{R}}{c}\)

\(\nabla\frac{1}{{{R}^{*}}} = \left(\frac{d}{d{{R}^{*}}}\frac{1}{{{R}^{*}}}\right)\left(\nabla {{R}^{*}}\right) = -\frac{1}{{{R}^{*}}^2}\nabla {{R}^{*}}\)

\(\nabla {{R}^{*}} = \left(\frac{\vec{r}}{{{R}^{*}}}\right)\left(1 + \frac{\vec{R}\vec{a}}{c^2}-\frac{v^2}{c^2}\right)-\frac{\vec{v}}{c}\)

электрическое поле:

\(\vec{E} = -\nabla\varphi - \frac{1}{c}\frac{\partial}{\partial t}\vec{A} = \frac{q}{{{R}^{*}}^2}\left(\nabla {{R}^{*}}+\frac{\vec{v}}{c^2}\frac{\partial}{\partial t}{{R}^{*}} - \frac{\vec{a}r}{c^2}\right)\),
где \({{R}^{*}} = R - \frac{\vec{v}\vec{R}}{c}\)

\(\vec{E} = \frac{q}{{{R}^{*}}^2}\left(\nabla {{R}^{*}}+\frac{\vec{v}}{c^2}\frac{\partial}{\partial t}{{R}^{*}} - \frac{\vec{a}r}{c^2}\right)\)

\(\vec{E} = \frac{q}{{{R}^{*}}^2}\left(\left(-\frac{\vec{R}}{c {{R}^{*}}}\right)\left(-c - \frac{\vec{R}\vec{a}}{c} + \frac{v^2}{c}\right)-\frac{\vec{v}}{c}+\frac{\vec{v}}{c^2}\left(\frac{R}{{{R}^{*}}}\left(-c - \frac{\vec{R}\vec{a}}{c} + \frac{v^2}{c}\right) + c\right) - \frac{\vec{a}R}{c^2}\right)\)

\(\vec{E} = \frac{q}{{{R}^{*}}^3}\left(\left(\vec{R}-\frac{R}{c}\vec{v}\right)\left(1 + \frac{\vec{R}\vec{a}}{c^2} - \frac{v^2}{c^2}\right) - \vec{a}\frac{{{R}^{*}}R}{c^2}\right)\)

и магнитное поле:

\(\frac{\partial}{\partial x}\vec{A} = \frac{\partial}{\partial x}\frac{q\vec{v}}{c{{R}^{*}} }\)

\(\frac{\partial}{\partial x}\vec{A} = \frac{q}{c{R}^{*}}\frac{\partial}{\partial x} \vec{v} - \frac{q\vec{v}}{c{{R}^{*}}^2}\frac{\partial}{\partial x} {{R}^{*}}\)

\(\frac{\partial}{\partial x}\vec{A} = \frac{q}{c{{R}^{*}} }\left(-\frac{\vec{a}R_x}{c{{R}^{*}}}\right)- \frac{q\vec{v}}{c{{R}^{*}}^2}\left(\frac{R_x}{{{R}^{*}}}\left(1 +\frac{\vec{R}\vec{a}}{c^2} - \frac{v^2}{c^2}\right) - \frac{v_x}{c}\right)\)

\(\vec{B} = \nabla\times\vec{A} = -\left(\frac{q}{c^2 {{R}^{*}}^2}\right)\vec{R}\times\vec{a} - \frac{q}{{{R}^{*}}^3 c}\left(1 + \frac{\vec{R}\vec{a}}{c^2} - \frac{v^2}{c^2}\right)\vec{R}\times\vec{v}\)

\(\vec{B} = -\frac{q}{{{R}^{*}}^3}\left(\vec{R}\times\frac{\vec{v}}{c}\left(1 + \frac{\vec{R}\vec{a}}{c^2} - \frac{v^2}{c^2}\right) + \vec{R}\times\frac{\vec{a}{{R}^{*}}}{c^2}\right)\)

\(\vec{B} = \frac{\vec{R}\times\vec{E}}{R}\)

    \begin{tcolorbox}[breakable, size=fbox, boxrule=1pt, pad at break*=1mm,colback=cellbackground, colframe=cellborder]
\prompt{In}{incolor}{ }{\boxspacing}
\begin{Verbatim}[commandchars=\\\{\}]

\end{Verbatim}
\end{tcolorbox}

    Учитывая вышеприведенные выкладки \cite{rustot}, для градиента по
координатам наблюдения скалярного потенциала ЛВ можно привести

\(\nabla \varphi =\nabla \frac{q}{{{R}^{*}}}=-\frac{q}{{{R}^{*}}^{2}}\left\{ \frac{\overrightarrow{R}}{{{R}^{*}}}\left( 1+\frac{\overrightarrow{a}\cdot \overrightarrow{R}}{{{c}^{2}}}-\frac{\overrightarrow{v}\cdot \overrightarrow{v}}{{{c}^{2}}} \right)-\frac{\overrightarrow{v}}{c} \right\}\)

А для производной по времени наблюдения векторного потенциала

\(\frac{1}{c}\frac{\partial }{\partial t}\overrightarrow{A}=\frac{1}{c}\frac{\partial }{\partial t}\frac{q\overrightarrow{v}}{c{{R}^{*}}}=\frac{q}{{{R}^{*}}^{2}}\left\{ \frac{\overrightarrow{a}R}{{{c}^{2}}}-\frac{\overrightarrow{v}}{c}\left( \frac{R}{{{R}^{*}}}\left( -1-\frac{\overrightarrow{a}\cdot \overrightarrow{R}}{{{c}^{2}}}+\frac{\overrightarrow{v}\cdot \overrightarrow{v}}{{{c}^{2}}} \right)+1 \right) \right\}\)

Суммарное электрическое поле (сравнить с ЛЛ2 (63,8))

\[\vec{E} = \frac{q}{{{R}^{*}}^{3}}\left( \left(\vec{R}-\frac{R}{c}\vec{v} \right) \left(1 + \frac{\vec{R}\vec{a}}{c^2} - \frac{v^2}{c^2} \right) - \vec{a}\frac{{R}^{*}R}{c^2} \right)\]

    Далее при имплеменации полученных формул в программные коды используется
нормированный на единицу радиус Лиенара Вихерта
\(k=\left( 1-\frac{\overrightarrow{v}\cdot \overrightarrow{n}}{c} \right)\),
где \(\overrightarrow{n} = \frac{\overrightarrow{R}}{R}\). При переходе
от формул к програмным кодам используется соотношение
\({R}^{*} = k\cdot R\)

    Таким образом в полученном выражении для электрического поля

    \(\vec{E} = \frac{q}{{{R}^{*}}^3}\left(\left(\vec{R}-\frac{R}{c}\vec{v}\right)\left(1 + \frac{\vec{R}\vec{a}}{c^2} - \frac{v^2}{c^2}\right) - \vec{a}\frac{{{R}^{*}}R}{c^2}\right)\)

при переходе к програмным кодам произведено преобразование

\({R} \rightarrow r\)

\({R}^{*} \rightarrow k\cdot r\)

\(\vec{R} \rightarrow r\cdot \vec{n}\)

и получено используемое в программе выражение

\(\vec{E} = \frac{q}{k^3}\left(\left(\vec{n}-\frac{\vec{v}}{c}\right)\left(1 + \frac{\vec{r}\vec{a}}{c^2} - \frac{v^2}{c^2}\right) \frac{1}{r^2} - \frac{k}{r}\frac{\vec{a}}{c^2}\right)\)

    В полученном выражении для магнитного поля

    \(\vec{B} = \frac{\vec{r}\times\vec{E}}{r}\)

\(\vec{B} = -\frac{q}{{{R}^{*}}^3}\left(\vec{R}\times\frac{\vec{v}}{c}\left(1 + \frac{\vec{R}\vec{a}}{c^2} - \frac{v^2}{c^2}\right) + \vec{R}\times\frac{\vec{a}{{R}^{*}}}{c^2}\right)\)

При переходе от формул к програмным кодам произведено преобразование

\({R} \rightarrow r\)

\({R}^{*} \rightarrow k\cdot r\)

\(\vec{R} \rightarrow r\cdot \vec{n}\)

\(\vec{B} = -\frac{q}{k^3}\left( \left(\vec{n}\times{\vec{v}}\right) \left(1 + \frac{\vec{r}\vec{a}}{c^2} - \frac{v^2}{c^2}\right) \frac{1}{c\,r^2} + \left(\vec{n}\times\vec{a}\right) \frac{k}{r\,c^2}\right)\)

или покомпонентно

\(B_x = -\frac{q}{k^3}\left( \left( n_y\,v_z - n_z\,v_y \right) \left(1 + \frac{\vec{r}\vec{a}}{c^2} - \frac{v^2}{c^2}\right) \frac{1}{c\,r^2} + \left( n_y\,a_z - n_z\,a_y \right) \frac{k}{r\,c^2}\right)\)

\(B_y = -\frac{q}{k^3}\left( \left( n_z\,v_x - n_x\,v_z \right) \left(1 + \frac{\vec{r}\vec{a}}{c^2} - \frac{v^2}{c^2}\right) \frac{1}{c\,r^2} + \left( n_z\,a_x - n_x\,a_z \right) \frac{k}{r\,c^2}\right)\)

\(B_z = -\frac{q}{k^3}\left( \left( n_x\,v_y - n_y\,v_x \right) \left(1 + \frac{\vec{r}\vec{a}}{c^2} - \frac{v^2}{c^2}\right) \frac{1}{c\,r^2} + \left( n_x\,a_y - n_y\,a_x \right) \frac{k}{r\,c^2}\right)\)

    \begin{tcolorbox}[breakable, size=fbox, boxrule=1pt, pad at break*=1mm,colback=cellbackground, colframe=cellborder]
\prompt{In}{incolor}{ }{\boxspacing}
\begin{Verbatim}[commandchars=\\\{\}]

\end{Verbatim}
\end{tcolorbox}


    % Add a bibliography block to the postdoc
    
    
    
\end{document}
